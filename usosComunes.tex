\section {Using \textit{had}, \textit{have}, and \textit{has}}
\begin{itemize}
	\item \textbf{\textit{Have}}: Used with \textbf{I, you, we, they} in the \textbf{Present Perfect} tense to show actions that started in the past and continue in the present, or to describe experiences without a specific time.
	\begin{itemize}
		\item Example: \textit{They have completed the project.}
	\end{itemize}
	
	\item \textbf{\textit{Has}}: Used with \textbf{he, she, it} in the \textbf{Present Perfect} tense for similar uses as \textit{have} but with third-person singular subjects.
	\begin{itemize}
		\item Example: \textit{She has visited many countries.}
	\end{itemize}
	
	\item \textbf{\textit{Had}}: Used with all subjects in the \textbf{Past Perfect} tense to indicate an action that was completed before another past action.
	\begin{itemize}
		\item Example: \textit{He had left before they arrived.}
	\end{itemize}
\end{itemize}

\section{Uses of \textit{was} and \textit{were}}
\begin{itemize}
	\item Se usa \textbf{was} con los sujetos en singular (\textit{I, he, she, it}) 
	\begin{itemize}
		\item \textit{\textbf{I was} at the park yesterday.}
		\item \textit{\textbf{She was} happy with the results.}
	\end{itemize}
	
	\item Se usa \textbf{were} con los sujetos en plural (\textit{you, we, they})
	\begin{itemize}
		\item \textit{\textbf{They were} late to the meeting.}
		\item \textit{\textbf{You were} amazing in the play.}
	\end{itemize}
	
	\item En preguntas y oraciones negativas, se mantiene el mismo uso:
	\begin{itemize}
		\item Pregunta: \textit{Was she there?} / \textit{Were they invited?}
		\item Negación: \textit{I wasn't ready for the test.} / \textit{They weren't at the party.}
	\end{itemize}
\end{itemize}
