\section {Past participle}
El \textit{past participle} es la forma del verbo que se utiliza para construir los \textbf{tiempos compuestos} (\textit{perfect tenses}) y la \textbf{voz pasiva} (\textit{passive voice}). En español equivale al participio, que termina en \texttt{-ado} o \texttt{-ido}.

Ejemplos: \textit{gone} (ido), \textit{played} (jugado), \textit{won} (ganado).  

\begin{itemize}
	\item Se necesita el auxiliar \textbf{have} para formar los tiempos compuestos:  
	\begin{itemize}
		\item \textit{have done}
		\item \textit{had traveled}.
	\end{itemize}

	\item Se utiliza el auxiliar \textbf{be} para formar la voz pasiva:  
	\begin{itemize}
		\item \textit{was written}
		\item \textit{are asked}.
	\end{itemize}
\end{itemize}

\section {The passive voice}
La voz pasiva se utiliza cuando se quiere dar más importancia a la acción que a quien la realiza.

\textbf{Estructura:}  
\textit{sujeto + to be + past participle (3ra columna del verbo)}  

\textbf{Ejemplos:}  
\begin{itemize}
	\item \textbf{Presente:}
	\begin{itemize}
		\item \textit{The best pasta \textbf{is produced} in Italy.}
		\item \textit{The vegetables \textbf{aren't cooked} enough.}
	\end{itemize}
	\item \textbf{Pasado:}
	\begin{itemize}
		\item \textit{The restaurant \textbf{was given} an excellent review.}
		\item \textit{The cars \textbf{weren't made} in this country.}
	\end{itemize}
\end{itemize}

\section {If + past tense + would (Second conditional)}
El segundo condicional se usa para describir situaciones \textbf{hipotéticas o imaginarias} en el presente o futuro.

\textbf{Estructura:}  
\textit{if + pasado simple}, \textit{would + infinitivo (sin ``to'')}  

\textbf{Ejemplos:}  
\begin{itemize}
	\item \textit{\textbf{If} my country \textbf{was} richer, I \textbf{wouldn't want} to live abroad.}
	\item \textit{\textbf{If} we \textbf{found} a cure for malaria, millions of lives \textbf{would be} saved.}
	\item \textit{\textbf{If} people \textbf{didn't drive}, the air \textbf{would be} cleaner.}
	\item \textit{\textbf{If} we \textbf{used} solar power, \textbf{would} we \textbf{save} money?}
\end{itemize}

También se puede invertir el orden:  
\begin{itemize}
	\item \textit{There \textbf{would be} less rain \textbf{if} the world's forests \textbf{disappeared}.}
	\item \textit{You \textbf{wouldn't be} so tired \textbf{if} you \textbf{went} to bed earlier.}
\end{itemize}

\section {Phrasal verbs}
Los \textit{phrasal verbs} son estructuras verbales compuestas por un verbo y una segunda partícula, que puede ser un adjetivo, un adverbio o una preposición.

\begin{tabular}{|p{4cm}|p{4cm}|p{4cm}|p{4cm}|}
	\hline
	\textbf{Phrasal Verb} & \textbf{Traducción} & \textbf{Phrasal Verb} & \textbf{Traducción} \\ \hline
	Take off & Despegar & Look after & Cuidar de \\ \hline
	Give up & Rendirse & Turn on & Encender \\ \hline
	Break down & Averiarse & Turn off & Apagar \\ \hline
	Call off & Cancelar & Find out & Descubrir \\ \hline
	Pick up & Recoger & Run out of & Quedarse sin \\ \hline
	Put off & Posponer & Get up & Levantarse \\ \hline
	Look for & Buscar & Take out & Sacar \\ \hline
	Carry on & Continuar & Go out & Salir \\ \hline
	Set up & Configurar/Preparar & Sit down & Sentarse \\ \hline
	Get on & Subirse & Take over & Asumir el control \\ \hline
	Give in & Ceder & Check in & Registrarse \\ \hline
	Look forward to & Esperar con ansias & Break up & Romper (una relación) \\ \hline
	Bring up & Criar, mencionar & Hold on & Esperar \\ \hline
	Turn down & Rechazar & Pass away & Fallecer \\ \hline
	Move on & Avanzar & Call back & Devolver la llamada \\ \hline
	Check out & Pagar e irse & Drop off & Dejar (a alguien o algo) \\ \hline
	Get back & Regresar & Hang out & Pasar el rato \\ \hline
	Keep up & Mantener el ritmo & Run into & Encontrarse con alguien \\ \hline
	Work out & Resolver, ejercitarse & Cut off & Cortar, interrumpir \\ \hline
	Stand up & Levantarse & Look up to & Admirar \\ \hline
	Put on & Ponerse (ropa) & Take back & Retirar \\ \hline
	Show up & Aparecer & Keep on & Seguir haciendo algo \\ \hline
	Fall apart & Desmoronarse & Throw away & Tirar (algo) \\ \hline
	Bring back & Devolver & Wake up & Despertarse \\ \hline
\end{tabular}

