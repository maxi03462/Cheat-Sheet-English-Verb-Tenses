\section{Uses of the \textit{-ing} Form}
\begin{itemize}
	\item Cuando el verbo se usa como sujeto.
	\begin{itemize}
		\item \textit{\textbf{Making} an omelette is easy.}
		\item \textit{\textbf{Reading} books helps improve vocabulary.}
	\end{itemize}

	\item Después de ciertos verbos:
	
	En los verbos de gustos, siempre se usa el \textit{-ing}, salvo que esté acompañado de \textit{would}.
	
	\begin{itemize}
		\item \textit{I \textbf{would like} to travel abroad.} (No se usa \textbf{liking}.)
		\item \textit{She \textbf{would prefer} to stay at home.} (No se usa \textbf{preferring}.)
	\end{itemize}
	\begin{itemize}
		\item \textit{\textbf{Avoid shopping} in supermarkets.}
		\item \textit{She \textbf{enjoys painting} landscapes.}
	\end{itemize}
	
	\item Después de una preposición: preposición + \textit{-ing}.
	
	\begin{itemize}
		\item \textit{I'm interested in \textbf{trying} new kinds of food.}
		\item \textit{He left without \textbf{saying} goodbye.}
	\end{itemize}
\end{itemize}

\section{Uses of the Infinitive with \textit{to}}
\begin{itemize}
	\item Después de ciertos verbos: \textit{want to}, \textit{need to}, etc.
	\begin{itemize}
		\item \textit{She \textbf{needs to} finish her homework.}
	\end{itemize}

	\item Infinitivo + \textit{to} cuando damos razones para hacer algo.
	\begin{itemize}
		\item \textit{I'm here \textbf{to see} the manager.}
		\item \textit{She has studied a lot \textbf{to pass} this exam.}
	\end{itemize}
	
	\item Después de un adjetivo se usa \textit{to + infinitivo}.
	\begin{itemize}
		\item \textit{It's \textbf{nice to meet} you, Ms. Evans.}
		\item \textit{It's \textbf{difficult to understand} his handwriting.}
	\end{itemize}
\end{itemize}

\section{\textit{Used to}}
\begin{itemize}
	\item Para hablar de rutinas o hábitos del pasado.
	
	\begin{itemize}
		\item \textit{I \textbf{used to} write for my university magazine.}
		\item \textit{They \textbf{used to} play soccer every weekend.}
	\end{itemize}
	
	 
	\item En el pasado, con estructuras negativas o preguntas, se usa \textbf{use} sin la \textit{d}.
	
	\begin{itemize}
		\item \textit{\textbf{Did} she \textbf{use} to live in New York?}
		\item \textit{He \textbf{didn't} \textbf{use} to like coffee.}
	\end{itemize}
\end{itemize}